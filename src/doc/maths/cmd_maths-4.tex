\noindent\begin{tabularx}{\linewidth}{XXXX}
    \textbf{Command}                      & \textbf{location}                                               & \textbf{short desc.}                            & \textbf{Example}               \\
    \midrule
    % ————————————————————————————————————————————————————————————
    \mintinline{latex}{\indicatrice}      & \detokenize{fonctions_et_}\linebreak\detokenize{operateurs.tex} & indicatrice d'un ensemble                       & $\indicatrice{A}$              \\ \\

    \mintinline{latex}{\norme}            & \detokenize{fonctions_et_}\linebreak\detokenize{operateurs.tex} & norme d'un élément                              & $\norme{p}{x}$                 \\ \\

    \mintinline{latex}{\dist}             & \detokenize{fonctions_et_}\linebreak\detokenize{operateurs.tex} & distance issue d'une norme entre deux vecteurs  & $\dist{x}{y}$                  \\ \\

    \mintinline{latex}{\distnorme}        & \detokenize{fonctions_et_}\linebreak\detokenize{operateurs.tex} & distance issue d'une norme entre deux vecteurs  & $\distnorme{\infty}{x}{y}$     \\ \\

    \mintinline{latex}{\prodscal(*)}      & \detokenize{fonctions_et_}\linebreak\detokenize{operateurs.tex} & produit scalaire entre deux vecteurs            & $\prodscal{x}{y}$              \\ \\

    \mintinline{latex}{\prodscalselon(*)} & \detokenize{fonctions_et_}\linebreak\detokenize{operateurs.tex} & produit scalaire [spécifié] entre deux vecteurs & $\prodscalselon{x}{y}{\infty}$ \\ \\

    \mintinline{latex}{\argmax(\limits)}  & \detokenize{fonctions_et_}

    operateurs.tex                        & argmax                                                          & $\argmax\limits_{x \in E} f(x)$                                                  \\ \\

    \mintinline{latex}{\argmin(\limits)}  & \detokenize{fonctions_et_}

    operateurs.tex                        & argmin                                                          & $\argmin\limits_{x \in E} f(x)$                                                  \\ \\

    \mintinline{latex}{\inverse}          & \detokenize{fonctions_et_}

    operateurs.tex                        & inverse d'un élément                                            & $\inverse{A}$                                                                    \\ \\

    \mintinline{latex}{\isdef}            & \detokenize{fonctions_et_}

    operateurs.tex                        & est défini comme                                                & $A \isdef B$                                                                     \\ \\

    \mintinline{latex}{\comm}             & \detokenize{fonctions_et_}

    operateurs.tex                        & commutant d'un ensemble d'opérateurs                            & $\comm{A}$                                                                       \\ \\

    \mintinline{latex}{\rg}               & \detokenize{fonctions_et_}

    operateurs.tex                        & rang d'un élément                                               & $\rg{A}$                                                                         \\ \\

    \mintinline{latex}{\im}               & \detokenize{fonctions_et_}

    operateurs.tex                        & image d'un élément                                              & $\im{A}$                                                                         \\ \\

    \mintinline{latex}{\pgcd}             & \detokenize{fonctions_et_}

    operateurs.tex                        & pgcd                                                            & $\pgcd{p}{q}$                                                                    \\ \\

    \mintinline{latex}{\positive}         & \detokenize{fonctions_et_}

    operateurs.tex                        & partie positive d'un élément                                    & $\positive{x^3 - x^2}$                                                           \\ \\
\end{tabularx}