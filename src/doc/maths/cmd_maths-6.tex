\noindent\begin{tabularx}{\linewidth}{XXXX}
    \toprule                                                                                            \\
    % Lebesgue
    \mintinline{latex}{\leb} & \detokenize{integral.tex} & Intégrale de Lebesgue (symbol différenciel) & $\leb$ \\ \\
    \mintinline{latex}{\lebesgue} & \detokenize{integral.tex} & Intégrale de Lebesgue $\oplus$ ensemble   & $\lebesgue {\mathds X}$ \\ \\
    \mintinline{latex}{\lebint} & \detokenize{integral.tex} & Intégrale de Lebesgue $\oplus$ de $a$ à $b$  & $\lebint a b$ \\ \\
    \mintinline{latex}{\lebm} & \detokenize{integral.tex} & Intégrale de Lebesgue (ensemble $\oplus$ intégrande $\oplus$ mesure )  & $\lebm {\mathds X} f \mu$ \\ \\
    % Bochner
    \midrule
    \mintinline{latex}{\boch} & \detokenize{integral.tex} & Intégrale de Bochner (symbol différenciel) & $\boch$ \\ \\
    \mintinline{latex}{\bochner} & \detokenize{integral.tex} & Intégrale de Bochner $\oplus$ ensemble   & $\bochner {\mathds X}$ \\ \\
    \mintinline{latex}{\bochint} & \detokenize{integral.tex} & Intégrale de Bochner $\oplus$ de $a$ à $b$  & $\bochint a b$ \\ \\
    \mintinline{latex}{\bochm} & \detokenize{integral.tex} & Intégrale de Bochner (ensemble $\oplus$ intégrande $\oplus$ mesure )  & $\bochm {\mathds X} f \mu$ \\ \\
    \midrule
\end{tabularx}
\noindent\begin{tabularx}{\linewidth}{XXXX}
    \midrule \\
    % Riemann
    \mintinline{latex}{\riem} & \detokenize{integral.tex} & Intégrale de Riemann (symbol différenciel) & $\riem$ \\ \\
    \mintinline{latex}{\riemann} & \detokenize{integral.tex} & Intégrale de Riemann $\oplus$ ensemble   & $\riemann {\mathds X}$ \\ \\
    \mintinline{latex}{\riemint} & \detokenize{integral.tex} & Intégrale de Riemann $\oplus$ de $a$ à $b$  & $\riemint a b$ \\ \\
    \mintinline{latex}{\riemm} & \detokenize{integral.tex} & Intégrale de Riemann (ensemble $\oplus$ intégrande $\oplus$ mesure )  & $\riemm {\mathds X} f \mu$ \\ \\
    % Pettis
    \midrule
    \mintinline{latex}{\pet} & \detokenize{integral.tex} & Intégrale de Pettis (symbol différenciel) & $\pet$ \\ \\
    \mintinline{latex}{\pettis} & \detokenize{integral.tex} & Intégrale de Pettis $\oplus$ ensemble   & $\pettis {\mathds X}$ \\ \\
    \mintinline{latex}{\petint} & \detokenize{integral.tex} & Intégrale de Pettis $\oplus$ de $a$ à $b$  & $\petint a b$ \\ \\
    \mintinline{latex}{\petm} & \detokenize{integral.tex} & Intégrale de Pettis (ensemble $\oplus$ intégrande $\oplus$ mesure )  & $\petm {\mathds X} f \mu$ \\ \\
    \bottomrule
\end{tabularx}