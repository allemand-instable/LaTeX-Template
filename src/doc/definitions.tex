\noindent\begin{tabularx}{\linewidth}{X}
    \toprule
    \textbf{definition/\faAsterisk}                                                                   \\
    \midrule
    \textbf{Description}                                                                              \\
    The commands associated with symbols and other things for mathematics / mathematical environments \\
    \midrule
\end{tabularx}
\noindent\begin{tabularx}{\linewidth}{XXXX}
    % \mintinline{latex}{} & \detokenize{.tex} & & $$ \\ \\
    \mintinline{latex}{\sssection} & \detokenize{redefine.tex}      & sous-sous section avec un carré au début              &  \\ \\
        \midrule

        % \mintinline{latex}{} & \detokenize{theorem_style.tex} & & $$ \\ \\
    \mintinline{latex}{\thm(*)}    & \detokenize{theorem_style.tex} & environnement théorème (* : non numéroté)             &  \\ \\
    \mintinline{latex}{\prop(*)}   & \detokenize{theorem_style.tex} & environnement proposition (* : non numéroté)          &  \\ \\
    \mintinline{latex}{\lem(*)}    & \detokenize{theorem_style.tex} & environnement lemme (* : non numéroté)                &  \\ \\
    \mintinline{latex}{\cor(*)}    & \detokenize{theorem_style.tex} & environnement corollaire (* : non numéroté)           &  \\ \\
    \mintinline{latex}{\exo(*)}    & \detokenize{theorem_style.tex} & environnement exercice (* : non numéroté)             &  \\ \\
    \mintinline{latex}{\rem(*)}    & \detokenize{theorem_style.tex} & environnement remarque (* : non numéroté)             &  \\ \\

    \midrule

    \verb|weierstrass|             & \detokenize{pgfplot.tex}       & Fonction de weierstrass à utiliser dans un plot LaTex &    \\
\end{tabularx}

\ifnum\value{code}=1
    \begin{minted}[linenos=true, frame=single, breaklines, breakanywhere,highlightlines=3]{latex}
\begin{tikzpicture}
    \begin{axis}
        \addplot [flatuicolors_green, samples=800, domain=0:1.1] {weierstrass(2*x,2,15)};
    \end{axis}
\end{tikzpicture}
\end{minted}
\fi

\noindent\begin{tabularx}{\linewidth}{XXXX}
    \midrule
\end{tabularx}

\noindent\begin{tabularx}{\linewidth}{XXXX}
    \midrule
    \mintinline{latex}{\lorem}     & \detokenize{lorem.tex}      & lorem ipsum placeholder text      & \lorem \\ \\
    \midrule
    % \mintinline{latex}{} & \detokenize{.tex} & & $$ \\ \\
    \mintinline{latex}{checkmarks} & \detokenize{checkmarks.tex} & checkmarks \linebreak environment &        \\
    \mintinline{latex}{\cmark}     & \detokenize{checkmarks.tex} & checkmark character               & \cmark \\
    \mintinline{latex}{\xmark}     & \detokenize{checkmarks.tex} & cross character                   & \xmark \\
    \mintinline{latex}{\checked}   & \detokenize{checkmarks.tex} & check the box                     &        \\
    \mintinline{latex}{\crossed}   & \detokenize{checkmarks.tex} & cross the box                     &        \\
\end{tabularx}

\begin{minted}[linenos=true, frame=single, breaklines, breakanywhere]{latex}
\begin{checkmarks}
    \item duh
    \item[\checked] checked
    \item[\crossed] crossed
\end{checkmarks}   
\end{minted}

\begin{checkmarks}
    \item duh
    \item[\checked] checked
    \item[\crossed] crossed
\end{checkmarks}

\noindent\begin{tabularx}{\linewidth}{XXXX}
    \midrule
    \mintinline{latex}{circledenum} & \detokenize{checkmarks.tex} & circledenum \linebreak environment & \\
\end{tabularx}

\begin{minted}[linenos=true, frame=single, breaklines, breakanywhere]{latex}
\begin{circledenum}
    \item le un
    \item le deux
    \item le trois
\end{circledenum}
\end{minted}

\begin{circledenum}
    \item le un
    \item le deux
    \item le trois
\end{circledenum}
