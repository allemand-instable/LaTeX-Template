% & Name
% 		weierstrass
% & Description
% 		Weierstrass function for pgf plot
% & Arguments
%		todo : savoir ce que font les parametres 2 et 3
% 		1: 𝑥
%		2: a
%		3: b
% & Examples
% 		courbe de weierstrass dans un tikzplot entre 0 et 1 
% 		\addplot [flatuicolors_green, samples=800, domain=0:1.1] {weierstrass(2*x,2,15)};
\makeatletter
\pgfmathdeclarefunction{weierstrass}{3}{%
	\pgfmathfloattofixed@{#3}%
	\afterassignment\pgfmath@x%
	\expandafter\c@pgfmath@counta\pgfmathresult pt\relax%
	\pgfmathfloatcreate{1}{0.0}{0}%
	\let\pgfmathfloat@loc@TMPa=\pgfmathresult
	\pgfmathfloatpi@%
	\let\pgfmathfloat@loc@TMPd=\pgfmathresult%
	\edef\pgfmathfloat@loc@TMPb{#1}%
	\edef\pgfmathfloat@loc@TMPc{#2}%
	\pgfmathloop
	\ifnum\c@pgfmath@counta>0\relax%
	\pgfmathfloatparsenumber{\the\c@pgfmath@counta}%
	\pgfmathpow{\pgfmathresult}{\pgfmathfloat@loc@TMPc}%
	\pgfmathfloatmultiply@{\pgfmathresult}{\pgfmathfloat@loc@TMPd}%
	\let\pgfmathfloat@loc@TMPe=\pgfmathresult%
	\pgfmathmultiply{\pgfmathresult}{\pgfmathfloat@loc@TMPb}%
	\pgfmathdeg{\pgfmathresult}%
	\pgfmathsin{\pgfmathresult}%
	\pgfmathdivide{\pgfmathresult}{\pgfmathfloat@loc@TMPe}%
	\pgfmathadd{\pgfmathresult}{\pgfmathfloat@loc@TMPa}%
	\let\pgfmathfloat@loc@TMPa=\pgfmathresult
	\advance\c@pgfmath@counta by-1\relax%
	\repeatpgfmathloop%
}
\makeatother