% & Name
% 		\intervaleint
% & Description
% 		Interval notation with square brackets ⟦p,q⟧
% & Arguments
% 		1: lower bound
% 		2: upper bound
% & Examples
% 		\intervaleint{p}{q}
\newcommand{\intervaleint}[2]{\llbracket #1 \, , \, #2 \rrbracket}

% & Name
% 		\R
% & Description
% 		Espace ℝᴾ
% & Arguments
% 		1: dimension
% & Examples
% 		\R{p}
%       \R p
\newcommand{\R}[1]{\mathds{R} ^{#1}}

% & Name
% 		\classespace
% & Description
% 		Espace des fonctions de classe k sur un ensemble E : 𝒞ᵏ(E)
% & Arguments
% 		1: classe
% 		2: ensemble
% & Examples
% 		\classespace{k}{E}
\newcommand{\classespace}[2]{\mathcal C^{#1} \left( \, #2 \, \right)}%

% & Name
% 		\continuborne
% & Description
% 		Espace des fonctions continues et bornées sur un ensemble E dans F : 𝒞⁰ᵇ(E, F)
% & Arguments
% 		1: ensemble de départ
% 		2: espace d'arrivée
% & Examples
% 		\continuborne{E}{F}
\newcommand{\continuborne}[2]{\mathcal C^0_b \left( \, #1 \,, \, #2 \, \right)}%

% & Name
% 		\continusupportcompact
% & Description
% 		Espace des fonctions continues à support compact sur un ensemble E dans F : 𝒞⁰ₖ(E, F)
% & Arguments
% 		1: ensemble de départ
% 		2: espace d'arrivée
% & Examples
% 		\continusupportcompact{E}{F}
\newcommand{\continusupportcompact}[2]{\mathcal C^0_K \left( \, #1 \,, \, #2 \, \right)}% 

% & Name
% 		\mesurable
% & Description
% 		Espace des fonctions mesurables sur un ensemble E dans F : 𝓜(E, F)
% & Arguments
% 		1: ensemble de départ
% 		2: espace d'arrivée
% & Examples
% 		\mesurable{E}{F}
\newcommand{\mesurable}[2]{{m}\left( \, #1 \,, \, #2 \, \right)}%

% & Name
% 		\etageepositive
% & Description
% 		Espace des fonctions étagées positives sur un ensemble E dans F : 𝓔₊(E, F)
% & Arguments
% 		1: ensemble de départ
% 		2: espace d'arrivée
% & Examples
% 		\etageepositive{E}{F}
\newcommand{\etageepositive}[2]{\mathcal E_+\left( \, #1 \,, \, #2 \, \right)}%

% & Name
% 		\VA
% & Description
% 		Variable aléatoire à valeur dans un espace E : 𝕍A[E]
% & Arguments
% 		1: espace d'arrivée
% & Examples
% 		\VA{E}
\newcommand{\VA}[1]{\mathds VA \bigl[ \, #1 \, \bigr]}%

% & Name
% 		\matrixspace
% & Description
% 		Espace des matrices carrées de taille n sur un espace E : 𝓜ₙ(E)
% & Arguments
% 		1: taille
% 		2: espace d'arrivée
% & Examples
% 		\matrixspace{n}{E}
\newcommand{\matrixspace}[2]{\mathcal M_{#1} \left( \, #2 \,  \right)}%

% & Name
% 		\orthonormal
% & Description
% 		symbole orthonormal : perp + norme en dessous
% & Arguments
% 		none
% & Examples
% 		\orthonormal
\newcommand{\orthonormal}{\underset{\Vert \cdot \Vert}{\perp}}

% & Name
% 		\orthonormalselon
% & Description
% 		symbole orthonormal : perp + norme [spécifiée] en dessous
% & Arguments
% 		1: subscript de la norme
% & Examples
% 		\orthonormalselon{\infty}
\newcommand{\orthonormalselon}[1]{\underset{\Vert \cdot \Vert_{#1}}{\perp}}


% & Name
% 		\grand[Lettre Maj]
% & Description
% 		lettre double stroke
% & Arguments
% 		none
% & Examples
% 		\grandR
%       \grandQ
%       \grandN
%       \grandH
\newcommand{\grandR}{\mathds{R}}%
\newcommand{\grandZ}{\mathds{Z}}%
\newcommand{\grandQ}{\mathds{Q}}%
\newcommand{\grandN}{\mathds{N}}%
\newcommand{\grandH}{\mathds{H}}%
\newcommand{\grandW}{\mathds{W}}%
\newcommand{\grandT}{\mathds{T}}%
\newcommand{\grandJ}{\mathds{J}}%
\newcommand{\grandL}{\mathds{L}}%
\newcommand{\grandP}{\mathds{P}}%
\newcommand{\grandK}{\mathds{K}}%
\newcommand{\grandX}{\mathds{X}}%
\newcommand{\grandY}{\mathds{Y}}%
\newcommand{\grandB}{\mathds{B}}%
\newcommand{\grandM}{\mathds{M}}%
\newcommand{\grandA}{\mathds{A}}%
\newcommand{\grandF}{\mathds{F}}%
\newcommand{\grandI}{\mathds{I}}%
\newcommand{\grandD}{\mathds{D}}%
\newcommand{\grandV}{\mathds{V}}%
\newcommand{\grandE}{\mathds{E}}%


% & Name
% 		\cal[Lettre Maj]
% & Description
% 		lettre calligraphiée
% & Arguments
% 		none
\newcommand{\calF}{\mathcal{F}}%
\newcommand{\calO}{\mathcal{O}}%
\newcommand{\calL}{\mathcal{L}}%
\newcommand{\calP}{\mathcal{P}}%
\newcommand{\calM}{\mathcal{M}}%
\newcommand{\calN}{\mathcal{N}}%
\newcommand{\calA}{\mathcal{A}}%
\newcommand{\calB}{\mathcal{B}}%
\newcommand{\calC}{\mathcal{C}}%
\newcommand{\calD}{\mathcal{D}}%
\newcommand{\calE}{\mathcal{E}}%
\newcommand{\calG}{\mathcal{G}}%
\newcommand{\calH}{\mathcal{H}}%
\newcommand{\calI}{\mathcal{I}}%
\newcommand{\calJ}{\mathcal{J}}%
\newcommand{\calK}{\mathcal{K}}%
\newcommand{\calQ}{\mathcal{Q}}%
\newcommand{\calR}{\mathcal{R}}%


% & Name
% 		\Rplus(etoile) / \Rmoins(etoile)
% & Description
% 		Ensembles ℝ₊ et ℝ₋
% & Arguments
% 		none
% & Examples
% 		\Rplus
%       \Rmoins
\newcommand{\Rplus}{\mathds R_{+}}%
\newcommand{\Rmoins}{\mathds R_{-}}%
\newcommand{\Rplusetoile}{\mathds R_{+}^{*}}%
\newcommand{\Rmoinsetoile}{\mathds R_{-}^{*}}%
\newcommand{\Retoile}{\mathds R^{*}}%
