%%  VA spéciales
% & Name
%       \statrang
% & Description
%       Xₙ⁽ᴵ⁾ : statistique d'ordre
%       i eme element dans l'ordre croissant du n-échantillon (Xₖ  k ∈ [1,n])
% & Arguments
%       1: Variable Aléatoire
%       2: nombre d'observations de l'échantillon
%       3: ordre
% & Examples
%       \statrang X n k
%       donne : Xₙ⁽ᵏ⁾
\newcommand{\statrang}[3]{{#1} _ {#2} ^{\left(#3\right)}}
%%  suites
% & Name
%       
% & Description
%       
% & Arguments
%       
% & Examples
%       
\newcommand{\suiteensemble}[1]{\left(  #1  \right)^\mathbb N}

% & Name
%       
% & Description
%       
% & Arguments
%       
% & Examples
%       
\newcommand{\suite}[2]{( {#1}_{#2})_{#2 \geq 0}}

% & Name
%       
% & Description
%       
% & Arguments
%       
% & Examples
%       
\newcommand{\soussuite}[2]{( {#1}_{n_{#2}})_{#2 \geq 0}}

% & Name
%       
% & Description
%       
% & Arguments
%       
% & Examples
%       
\newcommand{\famille}[2]{( {#1}_{#2})_{#2 \in I}}

% & Name
%       
% & Description
%       
% & Arguments
%       
% & Examples
%       
\newcommand{\suitecomposition}[3]{\left( #1( {#2}_{#3}) \right)_{#3 \geq 0}}

% & Name
%       
% & Description
%       
% & Arguments
%       
% & Examples
%       
\newcommand{\suitestatrang}[3]{ \left( \statrang{#1}{#3}{i} \right)_{#2, #3}}

% & Name
%       
% & Description
%       
% & Arguments
%       
% & Examples
%       
\newcommand{\famfinie}[4][i]{\left( \, {#2} _ {#1} \, \right)_{#3, #4}}

% & Name
%       
% & Description
%       
% & Arguments
%       
% & Examples
%       
\newcommand{\fromto}[3]{{#1}_{#2\, : \, #3}}

% & Name
%       
% & Description
%       
% & Arguments
%       
% & Examples
%       
\newcommand{\ordered}[2]{ {#1}_{(#2)} }
