% & Name
% 		\indicatrice
% & Description
% 		Indicatrice d'un ensemble noté sous la forme 𝕀_E
% & Arguments
% 		1: ensemble
% & Examples
% 		\indicatrice{E}
\newcommand{\indicatrice}[1]{{\mathds{1}}_{#1}}

% & Name
% 		\norme
% & Description
% 		Norme [spécifiée] d'un vecteur
% & Arguments
% 		1: subscript de la norme
% 		2: vecteur
% & Examples
% 		\norme{\infty}{x}
\newcommand{\norme}[2]{\left\lVert \, {#2} \, \right\rVert _{#1}}

% & Name
% 		\dist
% & Description
% 		Distance issue d'une norme entre deux vecteurs : ‖ x - y ‖
% & Arguments
% 		1: vecteur 1
% 		2: vecteur 2
% & Examples
% 		\dist{x}{y}
\newcommand{\dist}[2]{ \ \lVert {#1} - {#2} \rVert \ }

% & Name
% 		\distnorme
% & Description
% 		Distance [spécifiée] issue d'une norme entre deux vecteurs : ‖ x - y ‖_∞
% & Arguments
% 		1: subscript de la norme
% 		2: vecteur 1
% 		3: vecteur 2
% & Examples
% 		\distnorme{\infty}{x}{y}
\newcommand{\distnorme}[3]{ \norme{#1}{{#2} - {#3}}}

% & Name
% 		\prodscal(*)
% & Description
% 		Produit scalaire entre deux vecteurs ⟨x|y⟩
% & Arguments
%     *: uses bigl and bigr instead of left and right
% 		1: vecteur 1
% 		2: vecteur 2
% & Examples
% 		\prodscal{x}{y}
\NewDocumentCommand{\prodscal}{s m m}
{
  \IfValueTF{#1}{%
    \bigl\langle \, {#2} \, \vert \, {#3} \, \bigr\rangle
  }{%
    \left\langle \, {#2} \, \vert \, {#3} \, \right\rangle
  }%
}
% & Name
% 		\prodscalselon(*)
% & Description
% 		Produit scalaire [spécifié] entre deux vecteurs ⟨x|y⟩_∞
% & Arguments
%     *: uses bigl and bigr instead of left and right
% 		1: vecteur 1
% 		2: vecteur 2
% 		3: subscript du produit scalaire
% & Examples
% 		\prodscalselon{x}{y}{\infty}

% \newcommand*{\prodscalselon}[3]{\prodscal{#2}{#3}_{#4}}

\NewDocumentCommand{\prodscalselon}{s m m m}{
  \IfValueTF{#1}
  {%
    \prodscal*{#2}{#3}_{#4}
  }{%
    \prodscal{#2}{#3}_{#4}
  }%
}

% & Name
% 		\argmax
% & Description
% 		Argument maximum d'une fonction
% & Arguments
% 		none
% & Examples
% 		\argmax\limits_{u \in E}
\newcommand{\argmax}{\operatornamewithlimits{argmax}}
\newcommand{\argmin}{\operatornamewithlimits{argmin}}

% & Name
% 		\inverse
% & Description
% 		Inverse d'un élément : A⁻¹
% & Arguments
% 		1: élément
% & Examples
% 		\inverse{A}
\newcommand{\inverse}[1]{#1 ^ {-1}}


\newcommand{\isdef}{\underset {\textsf{déf}} {\equiv}}

% & Name
% 		\comm
% & Description	
% 		Commutant d'un ensemble d'opérateurs
% & Arguments
% 		1: ensemble d'opérateurs
% & Examples
% 		\comm{A}
\newcommand{\comm}[1]{\operatorname{Comm}\left( #1 \right)}

% & Name
% 		\rg
% & Description	
% 		Rang d'un opérateur / application
% & Arguments	
% 		1: opérateur / application
% & Examples	
% 		\rg{\Chi_A}
\newcommand{\rg}[1]{\operatorname{rg}\left( #1 \right)}
\newcommand{\im}{\operatorname{Im}}

% & Name
% 		\pgcd
% & Description
% 		Plus grand commun diviseur : pgcd(a, b)
% & Arguments
% 		1: premier nombre
% 		2: second nombre
% & Examples
% 		\pgcd{a}{b}
\newcommand{\pgcd}[2]{\operatorname{pgcd}\left( \, #1 , \, #2 \, \right)}

\newcommand{\positive}[1]{\left[ \, #1 \, \right]_+}

% & Name
% 		\func
% & Description
% 		Notations standarde d'une application
%  		E → F
%  		x ↦ f(x)
% & Arguments
% 		1: ensemble de départ
% 		2: espace d'arrivée
% 		3: variable	
% 		4: image
% & Examples
% 		\func{E}{F}{x}{f(x)}
\newcommand{\func}[4]{%
	\begin{array}{ccc} #1 &\longrightarrow& #2 \\ #3 &\longmapsto& #4 \end{array}
}%


% ! Référence :
% 🌐 https://tex.stackexchange.com/a/54395
%				— egreg
% & Name
% 		\opnorm
% & Description
% 		Norme opérateur
%  		||| T |||
% & Arguments
% 		1: Opérateur
% & Examples
% 		\opnorm{T}
%		\opnorm[\big]{a}  % slightly larger
%		\opnorm[\Bigg]{a} % largest
%		\opnorm*{a}       % \left and \right
\makeatletter
\newcommand{\opnorm}{\@ifstar\@opnorms\@opnorm}
\newcommand{\@opnorms}[1]{%
  \left|\mkern-1.5mu\left|\mkern-1.5mu\left|
   #1
  \right|\mkern-1.5mu\right|\mkern-1.5mu\right|
}
\newcommand{\@opnorm}[2][]{%
  \mathopen{#1|\mkern-1.5mu#1|\mkern-1.5mu#1|}
  #2
  \mathclose{#1|\mkern-1.5mu#1|\mkern-1.5mu#1|}
}
\makeatother


% & Name
% 		\Sp(*)
% & Description
% 		Spectre d'un endomorphisme ou opérateur T : sp(T)
% & Arguments
%     *: sans les parenthèses
% 		1: application 𝓊
% 		optionnel [] : corps du spectre
% & Examples
% 		\Sp{u} : sp(u)
%     \Sp[K]{u} : sp_K(u)
%     \Sp* u : sp u
\NewDocumentCommand{\Sp}{s o m}{%
  \IfBooleanTF{#1}
  {%
    % Starred version: \sp*
    \IfValueTF{#2}
    {%
      % \Sp[K]{u} -> sp_K(u) 
      \operatorname{sp}_{#2} {#3}%
    }
    {%
      % \Sp{u} -> sp(u)
      \operatorname{sp} {#3}%
    }%
  }
  {%
    % Non-starred version: \sp
    \IfValueTF{#2}
    {%
      % \Sp[K]{u} -> sp_K(u) 
      \operatorname{sp}_{#2}\bigl({#3}\bigr)%
    }
    {%
      % \Sp{u} -> sp(u)
      \operatorname{sp}\bigl({#3}\bigr)%
    }%
  }%
}


% & Name
% 		\spvec(*)
% & Description
% 		Vecteurs propres d'un endomorphisme ou opérateur T : vec{sp}(T)
% & Arguments
%     *: sans les parenthèses
% 		1: application 𝓊
% 		optionnel [] : corps du spectre
% & Examples
% 		\spvec{u} : vec{sp}(u)
\NewDocumentCommand{\spvec}{s o m}{%
  \IfBooleanTF{#1}
  {%
    % Starred version: \spvec*
    \IfValueTF{#2}
    {%
      % \spvec*[K]{u} -> sp_K u 
      \overrightarrow{\operatorname{sp}}_{#2} {#3}%
    }
    {%
      % \sp*{u} -> sp u
      \overrightarrow{\operatorname{sp}} {#3}%
    }%
  }
  {%
    % Non-starred version: \sp
    \IfValueTF{#2}
    {%
      % \spvec[K]{u} -> sp_K(u) 
      \overrightarrow{\operatorname{sp}}_{#2}\bigl({#3}\bigr)%
    }
    {%
      % \spvec{u} -> sp(u)
      \overrightarrow{\operatorname{sp}}\bigl({#3}\bigr)%
    }%
  }%
}

% & Name
% 		\spvecortho(*)
% & Description
% 		Vecteurs propres orthonormés d'un endomorphisme ou opérateur T : vec{sp}[ ⟂/‖⋅‖ ](T)
% & Arguments
%     *: sans les parenthèses
% 		optionnel [] : corps du spectre
% 		1: application 𝓊
% 		optionnel [] : norme considérée
% & Examples
% 		\spvecortho[K]{u}[\infty] : vec{sp}_K[ ⟂/‖⋅‖∞ ](u)
\NewDocumentCommand{\spvecortho}{s o m o}{%
  \IfBooleanTF{#1}
  {%
    % Starred version: \spvec*
    \IfValueTF{#2}
    {%
      % \spvec*[K]{u} -> sp_K u 
      \IfValueTF{#4}
      {%
      \overrightarrow{\operatorname{sp}}_{#2}\bigl[\, \vcenter{\hbox{$\underset{\| \cdot \|_{#4}}{\perp}$}} \,\bigr] \, {#3}%
      }%
      {%
        \overrightarrow{\operatorname{sp}}_{#2}\bigl[\, \vcenter{\hbox{$\underset{\| \cdot \|{\perp}$}}} \,\bigr] \, {#3}%
      }%
    }
    {%
      % \sp*{u} -> sp u
      \IfValueTF{#4}{%
        \overrightarrow{\operatorname{sp}}\bigl[\, \vcenter{\hbox{$\underset{\| \cdot \|_{#4}}{\perp}$}} \,\bigr] \, {#3}%
      }{%
        \overrightarrow{\operatorname{sp}}\bigl[\, \vcenter{\hbox{$\underset{\| \cdot \|{\perp}$}}} \,\bigr] \,{#3}%
      }%
    }%
  }%
  {%
    % Non-starred version: \sp
    \IfValueTF{#2}
    {%
      % \spvec[K]{u} -> sp_K(u) 
      \IfValueTF{#4}{%
        \overrightarrow{\operatorname{sp}}\bigl[\, \vcenter{\hbox{$\underset{\| \cdot \|_{#4}}{\perp}$}} \,\bigr]\bigl({#3}\bigr)%
      }{%
        \overrightarrow{\operatorname{sp}}\bigl[\, \vcenter{\hbox{$\underset{\| \cdot \|{\perp}$}}} \,\bigr]\bigl({#3}\bigr)%
      }%
    }%
    {%
      % \spvec{u} -> sp(u)
      \IfValueTF{#4}{%
        \overrightarrow{\operatorname{sp}}\bigl[\, \vcenter{\hbox{$\underset{\| \cdot \|}{\perp}$}} \,\bigr]\bigl({#3}\bigr)%
      }{%
        \overrightarrow{\operatorname{sp}}\bigl[\, \vcenter{\hbox{$\underset{\| \cdot \|{\perp}$}}} \,\bigr]\bigl({#3}\bigr)%
      }%
    }%
  }%
}


% & Name
% 		\inlinefunc
% & Description
% 		Notations standarde d'une application en inline
%  		E → F : 𝑥 ↦ 𝑓(𝑥)
%     avec les variations suivantes
%         ▶ {E}[𝑥]{F}[𝑦]: E → F : 𝑥 ↦ 𝑦
%  		  > prio ensembles
%         ▶ {E}{F}      : E → F 
%         ▶ {E}[𝑥]{F}   : 𝑥 ∈ E → F
%         ▶ {E}{F}[𝑦]   : E → F ∋ 𝑦
%       > prio objets (*)
%         ▶ *{𝑥}{𝑦}     : 𝑥 ↦ 𝑦
%         ▶ *{𝑥}[E]{𝑦}  : E ∋ 𝑥 ↦ 𝑦 
%         ▶ *{𝑥}{𝑦}[F]  : 𝑥 ↦ 𝑦 ∈ F
% & Arguments
%         ▶ {E}[𝑥]{F}[𝑦]: E → F : 𝑥 ↦ 𝑦
%  		  > prio ensembles
%         ▶ {E}{F}      : E → F 
%         ▶ {E}[𝑥]{F}   : 𝑥 ∈ E → F
%         ▶ {E}{F}[𝑦]   : E → F ∋ 𝑦
%       > prio objets (*)
%         ▶ *{𝑥}{𝑦}     : 𝑥 ↦ 𝑦
%         ▶ *{𝑥}[E]{𝑦}  : E ∋ 𝑥 ↦ 𝑦 
%         ▶ *{𝑥}{𝑦}[F]  : 𝑥 ↦ 𝑦 ∈ F
% & Examples
% 		f : \inlinefunc{E}[x]{F}[y]
%  ⚠️ différent de : (priorité objet)
%     f : \inlinefunc*{x}[E]{y}[F]
\NewDocumentCommand{\inlinefunc}{s m o m o}{
    \IfValueTF{#1}{%
      % ~ star (*)
      % set as mandatory
      \IfValueTF{#3}{%
        \IfValueTF{#5}{%
          % !     *{x}[E]{y}[F]
          {#2} \rightarrow {#4} \,:\, {#3} \mapsto {#5}%
        }{%
          % !     *{x}[E]{y}
          {#3} \ni {#2} \mapsto {#4}%
        }%
      }{%
        \IfValueTF{#5}{%
          % !     *{x}{y}[F]
          {#2} \mapsto {#4} \in {#5}%
        }{%
          % !     *{x}{y}
          {#2} \mapsto {#4}%
        }%
      }%
    }{%
      % ~ non star
      % value as mandatory
      \IfValueTF{#3}{%
        \IfValueTF{#5}{%
          % !   {E}[x]{F}[y]
          {#3} \rightarrow {#5} \,:\, {#2} \mapsto {#4}%
        }{%
          % !   {E}[x]{F}
          {#3} \in {#2} \rightarrow {#4}%
        }%
      }{%
        \IfValueTF{#5}{%
          % !   {E}{F}[y]
          {#2} \rightarrow {#4} \ni {#5}%
        }{%
          % !   {E}{F}
          {#2} \rightarrow {#4}%
        }%
      }%
    }%
}