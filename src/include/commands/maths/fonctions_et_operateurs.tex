% & Name
% 		\indicatrice
% & Description
% 		Indicatrice d'un ensemble noté sous la forme 𝕀_E
% & Arguments
% 		1: ensemble
% & Examples
% 		\indicatrice{E}
\newcommand{\indicatrice}[1]{\mathds 1 _{#1}}

% & Name
% 		\norme
% & Description
% 		Norme [spécifiée] d'un vecteur
% & Arguments
% 		1: subscript de la norme
% 		2: vecteur
% & Examples
% 		\norme{\infty}{x}
\newcommand{\norme}[2]{\left\lVert \, #2 \, \right\rVert _ {#1}}

% & Name
% 		\dist
% & Description
% 		Distance issue d'une norme entre deux vecteurs : ‖ x - y ‖
% & Arguments
% 		1: vecteur 1
% 		2: vecteur 2
% & Examples
% 		\dist{x}{y}
\newcommand{\dist}[2]{ \ \lVert #1 - #2 \rVert \ }

% & Name
% 		\distnorme
% & Description
% 		Distance [spécifiée] issue d'une norme entre deux vecteurs : ‖ x - y ‖_∞
% & Arguments
% 		1: subscript de la norme
% 		2: vecteur 1
% 		3: vecteur 2
% & Examples
% 		\distnorme{\infty}{x}{y}
\newcommand{\distnorme}[3]{ \norme{#1}{#2 - #3}}

% & Name
% 		\prodscal
% & Description
% 		Produit scalaire entre deux vecteurs ⟨x|y⟩
% & Arguments
% 		1: vecteur 1
% 		2: vecteur 2
% & Examples
% 		\prodscal{x}{y}
\newcommand{\prodscal}[2]{\left\langle \, #1 \, \vert \, #2 \, \right\rangle}
% & Name
% 		\prodscalselon
% & Description
% 		Produit scalaire [spécifié] entre deux vecteurs ⟨x|y⟩_∞
% & Arguments
% 		1: vecteur 1
% 		2: vecteur 2
% 		3: subscript du produit scalaire
% & Examples
% 		\prodscalselon{x}{y}{\infty}

\newcommand{\prodscalselon}[3]{\prodscal{#1}{#2}_{#3}}

% & Name
% 		\argmax
% & Description
% 		Argument maximum d'une fonction
% & Arguments
% 		none
% & Examples
% 		\argmax\limits_{u \in E}
\newcommand{\argmax}{\operatornamewithlimits{argmax}}
\newcommand{\argmin}{\operatornamewithlimits{argmin}}

% & Name
% 		\inverse
% & Description
% 		Inverse d'un élément : A⁻¹
% & Arguments
% 		1: élément
% & Examples
% 		\inverse{A}
\newcommand{\inverse}[1]{#1 ^ {-1}}


\newcommand{\isdef}{\underset {\textsf{déf}} {\equiv}}

% & Name
% 		\comm
% & Description	
% 		Commutant d'un ensemble d'opérateurs
% & Arguments
% 		1: ensemble d'opérateurs
% & Examples
% 		\comm{A}
\newcommand{\comm}[1]{\operatorname{Comm}\left( #1 \right)}

% & Name
% 		\rg
% & Description	
% 		Rang d'un opérateur / application
% & Arguments	
% 		1: opérateur / application
% & Examples	
% 		\rg{\Chi_A}
\newcommand{\rg}[1]{\operatorname{rg}\left( #1 \right)}
\newcommand{\im}{\operatorname{Im}}

% & Name
% 		\pgcd
% & Description
% 		Plus grand commun diviseur : pgcd(a, b)
% & Arguments
% 		1: premier nombre
% 		2: second nombre
% & Examples
% 		\pgcd{a}{b}
\newcommand{\pgcd}[2]{\operatorname{pgcd}\left( \, #1 , \, #2 \, \right)}

\newcommand{\positive}[1]{\left[ \, #1 \, \right]_+}

% & Name
% 		\func
% & Description
% 		Notations standarde d'une application
%  		E → F
%  		x ↦ f(x)
% & Arguments
% 		1: ensemble de départ
% 		2: espace d'arrivée
% 		3: variable	
% 		4: image
% & Examples
% 		\func{E}{F}{x}{f(x)}
\newcommand{\func}[4]{%
	\begin{array}{ccc} #1 &\longrightarrow& #2 \\ #3 &\longmapsto& #4 \end{array}
}%


% ! Référence :
% 🌐 https://tex.stackexchange.com/a/54395
%				— egreg
% & Name
% 		\opnorm
% & Description
% 		Norme opérateur
%  		||| T |||
% & Arguments
% 		1: Opérateur
% & Examples
% 		\opnorm{T}
%		\opnorm[\big]{a}  % slightly larger
%		\opnorm[\Bigg]{a} % largest
%		\opnorm*{a}       % \left and \right
\makeatletter
\newcommand{\opnorm}{\@ifstar\@opnorms\@opnorm}
\newcommand{\@opnorms}[1]{%
  \left|\mkern-1.5mu\left|\mkern-1.5mu\left|
   #1
  \right|\mkern-1.5mu\right|\mkern-1.5mu\right|
}
\newcommand{\@opnorm}[2][]{%
  \mathopen{#1|\mkern-1.5mu#1|\mkern-1.5mu#1|}
  #2
  \mathclose{#1|\mkern-1.5mu#1|\mkern-1.5mu#1|}
}
\makeatother