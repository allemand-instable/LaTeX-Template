% & Name
% 		\indicatrice
% & Description
% 		Indicatrice d'un ensemble noté sous la forme 𝕀_E
% & Arguments
% 		1: ensemble
% & Examples
% 		\indicatrice{E}
\newcommand{\indicatrice}[1]{\mathds 1 _{#1}}

% & Name
% 		\norme
% & Description
% 		Norme [spécifiée] d'un vecteur
% & Arguments
% 		1: subscript de la norme
% 		2: vecteur
% & Examples
% 		\norme{\infty}{x}
\newcommand{\norme}[2]{\left\lVert \, #2 \, \right\rVert _ {#1}}

% & Name
% 		\dist
% & Description
% 		Distance issue d'une norme entre deux vecteurs : ‖ x - y ‖
% & Arguments
% 		1: vecteur 1
% 		2: vecteur 2
% & Examples
% 		\dist{x}{y}
\newcommand{\dist}[2]{ \ \lVert #1 - #2 \rVert \ }

% & Name
% 		\distnorme
% & Description
% 		Distance [spécifiée] issue d'une norme entre deux vecteurs : ‖ x - y ‖_∞
% & Arguments
% 		1: subscript de la norme
% 		2: vecteur 1
% 		3: vecteur 2
% & Examples
% 		\distnorme{\infty}{x}{y}
\newcommand{\distnorme}[3]{ \norme{#1}{#2 - #3}}


\newcommand{\prodscal}[2]{\left\langle \, #1 \, \vert \, #2 \, \right\rangle}
\newcommand{\prodscalselon}[3]{\prodscal{#1}{#2}_{#3}}

% \argmax\limits_{}
\newcommand{\argmax}{\operatornamewithlimits{argmax}}
\newcommand{\argmin}{\operatornamewithlimits{argmin}}

\newcommand{\inverse}[1]{#1 ^ {-1}}

\newcommand{\isdef}{\underset {\textsf{déf}} {\equiv}}
\newcommand{\comm}[1]{\operatorname{Comm}\left( #1 \right)}
\newcommand{\rg}[1]{\operatorname{rg}\left( #1 \right)}
\newcommand{\im}{\operatorname{Im}}
\newcommand{\pgcd}[2]{\operatorname{pgcd}\left( \, #1 , \, #2 \, \right)}

\newcommand{\positive}[1]{\left[ \, #1 \, \right]_+}

\newcommand{\func}[4]{%
	\begin{array}{ccc} #1 &\longrightarrow& #2 \\ #3 &\longmapsto& #4 \end{array}
}%