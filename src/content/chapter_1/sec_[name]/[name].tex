\lorem\footnote{oui c'est vrai}\lorem

Я рад видеть тебя, мой друг
\begin{equation}
    \mathds Y = m(x) + \varepsilon
\end{equation}

\begin{equation}
\mathit Я + \mathsf д = \left(\mathbf г \cdot \frac{\mathsf и}{\mathit ц  \times \mathit н - \mathsf ԕ \cdot \mathsf д}\right) + \mathsf Љ \mathsf щ \mathsf ш \mathsf ю \mathsf ц \mathsf Ч \mathsf Ԕ \equiv \mathbf Ж + Ж
\end{equation}

\begin{equation}
    \mathbf ж + ж + \mathit ж
\end{equation}

\begin{equation}
f : \func {\mathds E} {\mathds F} {\mathit Ж} {\mu(\mathit ж)\footnotemark}
\end{equation}
\footnotetext{oui, ça aussi c'est vrai.}

éà ù oui µ mais\footnote{technique ça.} £ ÀÈÉ ne veut pas dire que @²

\textbf{éà ù oui µ mais £ ÀÈÉ ne veut pas dire que @² : Я рад видеть тебя, мой друг}

\textit{éà ù oui µ mais £ ÀÈÉ ne veut pas dire que @² : Я рад видеть тебя, мой друг}

\underline{éà ù oui µ mais £ ÀÈÉ ne veut pas dire que @² : Я рад видеть тебя, мой друг} $\zhe + \ge - \efit + \tse \ya$