%%  notation convergence
% uses mathtools package
\newcommand{\cvl}[2]{\wrightarrow[#1 \rightarrow #2]{\ \mathcal L \ } \ }
\newcommand{\cvp}[2]{\xrightarrow[#1 \rightarrow #2]{\ \mathbb P \ } }
\newcommand{\cvps}[2]{\xrightarrow[#1 \rightarrow #2]{\ \textsf{p.s}\ } \ }
\newcommand{\cvL}[3]{\xrightarrow[#2 \rightarrow #3]{\ \mathbb L ^{#1} \ }}
\newcommand{\cvetr}[2]{\xrightarrow[#1 \rightarrow #2]{ \ \textsf{étroit.} \ }}
\newcommand{\cvnorme}[3]{\xrightarrow[\, #2 \rightarrow #3 \; ]{\, \lVert \cdot \rVert _{#1} \,} \ }
\newcommand{\cvpp}[3]{ \xrightarrow[#2 \rightarrow #3]{#1 - p.p} }
%%  autres
\newcommand{\tend}[2]{\xrightarrow[ #1 \rightarrow #2 ]{}}
\newcommand{\tendset}[3]{\xrightarrow[ #1 \rightarrow #2 ]{#3}}
